% Options for packages loaded elsewhere
\PassOptionsToPackage{unicode}{hyperref}
\PassOptionsToPackage{hyphens}{url}
%
\documentclass[
]{article}
\usepackage{amsmath,amssymb}
\usepackage{lmodern}
\usepackage{iftex}
\ifPDFTeX
  \usepackage[T1]{fontenc}
  \usepackage[utf8]{inputenc}
  \usepackage{textcomp} % provide euro and other symbols
\else % if luatex or xetex
  \usepackage{unicode-math}
  \defaultfontfeatures{Scale=MatchLowercase}
  \defaultfontfeatures[\rmfamily]{Ligatures=TeX,Scale=1}
\fi
% Use upquote if available, for straight quotes in verbatim environments
\IfFileExists{upquote.sty}{\usepackage{upquote}}{}
\IfFileExists{microtype.sty}{% use microtype if available
  \usepackage[]{microtype}
  \UseMicrotypeSet[protrusion]{basicmath} % disable protrusion for tt fonts
}{}
\makeatletter
\@ifundefined{KOMAClassName}{% if non-KOMA class
  \IfFileExists{parskip.sty}{%
    \usepackage{parskip}
  }{% else
    \setlength{\parindent}{0pt}
    \setlength{\parskip}{6pt plus 2pt minus 1pt}}
}{% if KOMA class
  \KOMAoptions{parskip=half}}
\makeatother
\usepackage{xcolor}
\usepackage[margin=1in]{geometry}
\usepackage{graphicx}
\makeatletter
\def\maxwidth{\ifdim\Gin@nat@width>\linewidth\linewidth\else\Gin@nat@width\fi}
\def\maxheight{\ifdim\Gin@nat@height>\textheight\textheight\else\Gin@nat@height\fi}
\makeatother
% Scale images if necessary, so that they will not overflow the page
% margins by default, and it is still possible to overwrite the defaults
% using explicit options in \includegraphics[width, height, ...]{}
\setkeys{Gin}{width=\maxwidth,height=\maxheight,keepaspectratio}
% Set default figure placement to htbp
\makeatletter
\def\fps@figure{htbp}
\makeatother
\setlength{\emergencystretch}{3em} % prevent overfull lines
\providecommand{\tightlist}{%
  \setlength{\itemsep}{0pt}\setlength{\parskip}{0pt}}
\setcounter{secnumdepth}{-\maxdimen} % remove section numbering
\usepackage{tikz}
\newcommand{\boxaroundn}{\tikz[baseline=(n.base)]{\node[inner sep=1pt,draw,rectangle] (n) {\( n \)};}}
\usepackage{actuarialsymbol}
\ifLuaTeX
  \usepackage{selnolig}  % disable illegal ligatures
\fi
\IfFileExists{bookmark.sty}{\usepackage{bookmark}}{\usepackage{hyperref}}
\IfFileExists{xurl.sty}{\usepackage{xurl}}{} % add URL line breaks if available
\urlstyle{same} % disable monospaced font for URLs
\hypersetup{
  pdftitle={Jens Kahlenberg},
  pdfauthor={Daniel Smith},
  hidelinks,
  pdfcreator={LaTeX via pandoc}}

\title{Jens Kahlenberg}
\author{Daniel Smith}
\date{2024-10-17}

\begin{document}
\maketitle

\hypertarget{intro}{%
\section{Intro}\label{intro}}

\hypertarget{kapitel-2---einordnung}{%
\subsection{Kapitel 2 - Einordnung}\label{kapitel-2---einordnung}}

\hypertarget{cash-flow-projection-model-kurz-cfpm}{%
\subsubsection{\texorpdfstring{\textbf{Cash Flow Projection Model}
(kurz:
\textbf{CFPM})}{Cash Flow Projection Model (kurz: CFPM)}}\label{cash-flow-projection-model-kurz-cfpm}}

Betrachtet man die Geldflüsse zum Zeitpunkt des Abschlusses eines
Lebensversicherungsvertrags, so stellt man fest, dass eine neue Police
für das Unternehmen wirtschaftlich eine Investition darstellt. Denn: Im
Geschäftsjahr, in dem die Police zugeht, bewirkt diese üblicherweise
einen Verlust, der darin begründet ist, dass bei Vertragsabschluss
Provisionen für die erfolgreiche Vertragsvermittlung gezahlt werden, die
eingehenden Beiträge der Kunden in ebendiesem Geschäftsjahr jedoch
i.d.R. deutlich niedriger als die Höhe der ausgezahlten Provisionen
sind.

\hypertarget{embedded-value-kurz-ev}{%
\subsubsection{\texorpdfstring{\textbf{Embedded Value} (kurz:
\textbf{EV})}{Embedded Value (kurz: EV)}}\label{embedded-value-kurz-ev}}

Definiert als die Summe aller künftigen Gewinne des Unternehmens,
ausgedrückt im heutigen Geld. --\textgreater{} \emph{Barwertbetrachtung}

\hypertarget{stochastische-bewertung}{%
\subsubsection{Stochastische Bewertung}\label{stochastische-bewertung}}

Barwertbetrachtung:\\
Unterstellt man eine risikofreie Zinsstrukturkurve (z.B. die jährlichen
Renditen Schweizer Staatanleihen), so kann man durch Diskontierung den
heutigen Wert einer Auszahlung \(K_0\) in \(n\) Jahren ermitteln. Was
diese Betrachtung jedoch nicht berücksichtigt, sind Schwankungen auf den
Kapitalmärkten und ihr Einfluss auf den Wert des Portfolios.

Weshalb sind Kapitalmarkschwankungen überhaupt relevant für die
Bewertung von Versicherungspolicen?\\
Die Art der Beziehung zwischen Versicherer und Versicherungsnehmer ist
asymmetrisch. Während der Kunde den Vertrag stornieren kann, kann dies
der Versicherer nicht. Der Kunde in vielen aufgeschobenen
Rentenversicherungen entscheiden, ob dieser am Ende der Aufschubsdauer
das erreichte Kapital ausbezahlt bekommt oder sich eine Leibrente
ausbezahlen möchte; der Versiherer wird dazu nicht gefragt. Ebenso kann
der Kunde einer planmässigen Beitragserhöhung wiedersprechen; und der
Versicherer muss die Entscheidung des Kunden akzeptieren. Diese
Asymmetrien laufen unter der Bezeichnung \textbf{Optionen und
Garantien}. --\textgreater{} verhält sich wie ein \emph{Finanzderivat}!
Googeln, wie Finanzderivat definiert ist!!!

Und ähnlich wie bei Derivaten erfolgen künftige Zahlungsströme nicht
vorherbestimmt (man sagt auch deterministisch), sondern sie sind
abhängig von einer oder mehreren schwankenden Größen wie Zinshöhen oder
Aktienkursen, deren Wert wiederum von der Entwicklung der Kapitalmärkte
abhängt.

für Lebensversicherer:\\
Die wesentlichen künftigen Zahlungsströme bestehen aus Prämien,
Leistungen, Kosten, Dividenden und Steuern!

Zwei Sichtweisen:\\
1. Aktionärssicht (Blickrichtung des Eigentümers)\\
--\textgreater{} Wert des Portfolios steht im Vordergrund\\
--\textgreater{} Barwert der (zukünftigen) Dividenden berechnen\\
--\textgreater{} Barwert gewöhnlich als \textbf{Present Value of Future
Profits} (kurz: \textbf{PVFP})

\begin{enumerate}
\def\labelenumi{\arabic{enumi}.}
\setcounter{enumi}{1}
\tightlist
\item
  Kundensicht\\
  --\textgreater{} Höhe der Verpflichtungen gegenüber Kunden im
  Vordergrund\\
  --\textgreater{} Barwert der Auszahlungen (Leistungen und Kosten)
  abzüglich der Einzahlungen (Prämien) berechnen\\
  --\textgreater{} Resultierendee Betrag: \textbf{Best Estimate (of)
  Liabilities} (kurz: \textbf{BEL})
\end{enumerate}

\textbf{stochastisches (Unternehmens-)Modell} (kurz: \textbf{SUM})\\
Auszahlungsmuster für Dividenden hochkomplex und Schwankungen der
Kapitalmärkte können nicht direkt abgeleitet werden, da viel zu komplex.
--\textgreater{} keine analytische Formel\\
Idee: \textbf{Monte-Carlo-Simulationen} basierend auf tausenden
SImulationen von Zinsen, Aktien und anderen Kapitalgrössen\\
Diesen Kapitalmarktgrössen liegen stochastische Differenzialgleichungen
zugrunde, welche die Dynamik von diesen beschreibt.

--\textgreater{} siehe Abschnitt 2.3. in {[}Jens{]} Komponenten eines
stochastischen Unternehmensmodellsür weitere Details!!

\hypertarget{solvency-2}{%
\subsubsection{Solvency 2}\label{solvency-2}}

\textbf{Solvency II} ist eine Sammlung von europaweiten Regelungen,
welche den Versicherungsunternehmen u.a. feste Vorgaben zur Berechnung
der Solvabilität macht.

\textbf{Solvabilität:} Verhältnis zwischen den verfügbaren Eigenmitteln
und demjenigen Verlust, der (modellmäßig) nur einmal in 200 Jahren
auftreten kann. Sind die Eigenmittel höher als der Verlust in einem
derart verheerenden Ereignis, kann das Unternehmen eine solch ungünstige
Situation aus eigener Kraft überstehen --- es ist also solvent.

Die BEL beschreibt den besten Schätzer (Englisch „best estimate``) für
die in den bestehenden Policen enthaltenen Zusagen bzw. Verpflichtungen
(Englisch „liabilities``) den Versicherungsnehmern gegenüber. Solvency
II fordert speziell von den Lebensversicherern, die BEL als Barwert der
Auszahlungen abzüglich des Barwertes der Einnahmen zu berechnen.

Seit der Einführung von Solvency II soll die Reserve marktnah ermittelt
werden. Daraus folgt, dass die Entwicklung der Kapitalanlagen, die
künftigen Überschusszuteilungen an Kunden und die möglichen Handlungen
der Versicherungsnehmer (bspw. Stornierung, Beitragsfreistellung,
Kapitalabfindung statt Rentenauszahlung) realistisch geschätzt werden.
Wie oben bereits dargelegt, erfolgt diese Schätzung mithilfe von vielen
Monte-Carlo-Simulationen, um im Mittel den besten Schätzer für die
Verpflichtungen zu gewinnen.

Die meisten Risiken aus der Solvency II-Risikolandkarte eines
Lebensversicherungsunternehmens werden mithilfe von stochastischen
Unternehmensmodellen bewertet: sowohl die Marktrisiken, d.h.
Aktienentwicklung, Immobilienentwicklung, Zinsbewegung, Zinsvolatilität,
Währungsschwankungen, Ausfall von Anleihen, als auch die
versicherungstechnischen Risiken Biometrie (Sterblichkeit einschließlich
Katastrophen, Langlebigkeit, Invalidität), Kosten und Storno.

\hypertarget{kapitel-3---grundlagen-der-lebensversicherung}{%
\subsection{Kapitel 3 - Grundlagen der
Lebensversicherung}\label{kapitel-3---grundlagen-der-lebensversicherung}}

Was ist Versicherung bzw. eine Versicherung?\\
--\textgreater{} Grundlegenden Begriffen vertraut gemacht und
wesentliche Details von Versicherungen vorgestellt

\hypertarget{risikotransfer}{%
\subsubsection{Risikotransfer}\label{risikotransfer}}

Versicherung stellen einen Risikotransfer dar.\\
Die wichtigsten, durch Lebensversicherungen gedeckten Risiken sind Tod
und Invalidität sowie das Langlebigkeitsrisiko. Daneben gibt es noch
Versicherungsschutz für Risiken wie Pflegebedürftigkeit oder den
Eintritt schwerer Krankheiten (Dread Disease).

Warum ist das Risiko beim Versicherungsunternehmen besser aufgehoben ist
als bei der einzelnen Person selbst?

\begin{itemize}
\tightlist
\item
  auf den ersten Blick: größere Finanzstärke\\
\item
  Hauptgrund: \textbf{mathematischer Grund --\textgreater{} sog.
  Risikoausgleich im Kollektiv}\\
  Die Tatsache, dass der Versicherer viele gleichartige Risiken
  übernimmt, führt nämlich zu Ausgleichseffekten in seinem Bestand an
  Versicherungen. Dies bedeutet, dass sich die zufällige Schwankung in
  der Höhe der zu zahlenden Schadenleistungen bei zunehmender Anzahl
  gleichartiger Verträge reduziert und somit stabiler, \textbf{für den
  Lebensversicherer besser kalkulierbar wird}.
\end{itemize}

\hypertarget{rechnungsgrundlagen}{%
\subsubsection{Rechnungsgrundlagen}\label{rechnungsgrundlagen}}

Die bestehende Auflage, Reserven in ausreichender Höhe zu stellen,
impliziert insbesondere, vorsichtige Annahmen über diejenigen Grössen zu
treffen, welche einen Einfluss auf die Entwicklung des Vertrags haben,
nämlich:

Zins,

Biometrie,

Kosten.

Diese drei Einflussfaktoren werden unter dem Begriff \textbf{Annahmen}
bzw. \textbf{Rechnungsgrundlagen} zusammengefasst.

\textbf{Rechnungsgrundlagen 1. Ordnung}\\
entsprechen den vorsichtigen Annahmen über die drei vorgenannten
Grössen. Sie werden für die Berechnung von Prämien und Reserven
verwendet, wobei es zwei voneinander abweichende Sätze von Annahmen für
die Prämien- und Reserveberechnung geben kann.\\
--\textgreater{} Rechnungsgrundlagen 1. Ordnung googlen!!

\textbf{Rechnungsgrundlagen 2. Ordnung}\\
spiegeln die tatsächliche Erfahrung des Versicherers wider bzw. das, was
der Versicherer als realistischen „Schadenverlauf`` erwartet.\\
Wie wir in Kap. 10 sehen werden, führen Unterschiede in den
Rechnungsgrundlagen 1. und 2. Ordnung zur Entstehung von Überschüssen,
welche ihrerseits den Versicherten wieder zugute kommen.\\
--\textgreater{} Rechnungsgrundlagen 2. Ordnung googlen und Link zu Kap
10 später einführen!!

\hypertarget{standard-versicherungen}{%
\subsubsection{Standard-Versicherungen}\label{standard-versicherungen}}

Im Folgenden werden die gängigsten Produkte der Lebensversicherung und
ihre Charakteristika beschrieben.

Unterscheidungskriterien:\\

\textbf{Versorgungsschicht}: siehe Abschnitt 3.3. --\textgreater{}
bezieht sich auf Deutschland, nicht Schweiz!

\textbf{Produkte} bzw. \textbf{Produktarten}: gibt es Produkte zur
Absicherung des Todesfallrisikos, des Erlebensfall- oder
Langlebigkeitsrisikos sowie des Invaliditätsrisikos

\textbf{Tarife} --\textgreater{} \textbf{Tarifstufe} oder
\textbf{Kostenstufe} --\textgreater{} \textbf{Tarifgeneration}

\textbf{Vertragsteil:} Hauptversicherung vs.~Zusatzversicherung

\textbf{Hauptversicherung:} rechtlich eigenständiger
Versicherungsvertrag -\textgreater{} bildet die Hauptkomponente einer
Versicherung

\textbf{Zusatzversicherung:} Anbündelung einer Zusatzkomponente an eine
Hauptversicherung dar, mit der ein zusätzlicher Versicherungsschutz
eingeschlossen wird. Eine Zusatzversicherung bildet eine rechtliche
Einheit mit der Hauptversicherung und kann i.d.R. allein, ohne die
Hauptversicherung, nicht fortbestehen. Dies bedeutet, dass bei
Beendigung der Hauptversicherung auch die Zusatzversicherung endet.
Umgekehrt kann eine Zusatzversicherung für sich allein jedoch durchaus
gekündigt werden, ohne dass die Hauptversicherung erlischt.

\textbf{Geschäftsart:} Einzelgeschäft oder Kollektivgeschäft

\hypertarget{risikolebensversicherung}{%
\paragraph{Risikolebensversicherung}\label{risikolebensversicherung}}

Die Risikolebensversicherung gewährt Versicherungsschutz ausschließlich
für den Todesfall. Nennt man häufig auch \textbf{reinen
Todesfallversicherung}.

Versicherungsart: Stirbt versicherte Person innerhalb der
Versicherungsdauer, so wird eine Todesfallleistung (=vereinbarte
Versicherungssumme) fällig. Bei einer lebenslangen
Risikolebensversicherung wird die Versicherungsleistung folglich in
jedem Fall ausgezahlt.

Vertragsteil: Risikolebensversicherungen können als Haupt- und als
Zusatzversicherungen abgeschlossen werden.

Tarife: Das Leistungsspektrum derartiger Versicherungen umfasst sowohl
Tarife mit gleichbleibender (konstanter) Versicherungssumme als auch
Tarife mit einer planmäßig steigenden und/oder fallenden
Versicherungssumme. Die Veränderung kann entweder linear, prozentual
oder nach einem anderen Muster erfolgen.

Differenzierungen nach Risikomerkmalen: Bei Risikolebensversicherungen
findet häufig eine Differenzierung nach zusätzlichen
sterblichkeitsbeeinflussenden Risikomerkmalen statt. So gibt es bspw.
Unterscheidungen nach dem Raucherstatus (Raucher oder Nichtraucher) oder
nach dem Akademiker-Status (Akademiker oder Nicht-Akademiker), die auch
in Kombination auftreten können und somit zu einer Aufsplittung in
homogenere Zielgruppen führen. Derartige Tarife werden auch als
\textbf{Preferred Lives-Tarife} bezeichnet, da die Prämien für die
bevorzugten Personengruppen (z.B. nicht rauchende Akademiker) deutlich
reduziert ausfallen.

Verbundene Leben: Risikolebensversicherungen können auch auf verbundene
Leben abgeschlossen werden. Bei Tod einer der versicherte Personen wird
die Versicherungssumme fällig, jedoch nur ein einziges Mal, d.h. wenn
die erste oder die letzte VP verstirbt. Derartige Versicherungen dienen
der gegenseitigen Absicherung der jeweils anderen versicherte Personen.

\hypertarget{kapitallebensversicherung}{%
\paragraph{Kapitallebensversicherung}\label{kapitallebensversicherung}}

\textbf{reine Erlebensfallversicherung:} Beim Erleben des Ablaufs der
Versicherungsdauer wird die Versicherungssumme auszahlt.\\
--\textgreater{} historisches Relikt, das in dieser Reinform heute kaum
noch zu finden ist

\textbf{gemischte Versicherung:} Kombination aus der reinen Todesfall-
und der reinen Erlebensfallversicherung:\\
Bei Tod innerhalb der Versicherungsdauer wird eine Todesfallleistung
fällig; überlebt die VP jedoch bis zum Ende der Versicherungsdauer, wird
eine Erlebensfallleistung ausgezahlt.

Beachte:\\
Klassisch stimmen die für den Todes- und den Erlebensfall versicherten
Summen überein. Angeboten werden inzwischen jedoch auch Varianten mit
voneinander abweichender Todes- und Erlebensfallsumme. Dabei kann das
Verhältnis dieser beiden Grössen konstant sein oder im Laufe der
Versicherungsdauer variieren. Bspw. kann die Todesfallsumme zu
Vertragsbeginn gegenüber der Erlebensfallsumme abgesenkt sein und gegen
Ende der Versicherungsdauer auf die Erlebensfallsumme ansteigen. Denkbar
ist auch, dass die Todesfallsumme anfänglich höher ist als die
Erlebensfallsumme und linear auf diese absinkt. Es lassen sich nahezu
beliebige Leistungsspektren konstruieren.

kapitalbildende Lebensversicherung vs Kapitallebensversicherung\\
Es sei angemerkt, dass der Begriff kapitalbildende Lebensversicherung
keineswegs gleichbedeutend mit dem der Kapitallebensversicherung ist.
Vielmehr ist ersterer ein \textbf{Oberbegriff für alle diejenigen
Lebensversicherungen, bei denen in nicht unerheblichem Maße ein Kapital
gebildet werden muss, um die als sehr wahrscheinlich einzustufende
Auszahlung der Versicherungsleistung durchführen zu können}. Demnach
gehören die Kapitallebensversicherung und die Rentenversicherung zu den
kapitalbildenden Versicherungen, während Risikolebens- und
Berufsunfähigkeitsversicherung nicht dazu zählen, sondern unter dem
Begriff Risikoversicherung zusammengefasst werden.

\hypertarget{termfix-versicherung}{%
\subsubsection{Termfix-Versicherung}\label{termfix-versicherung}}

Abschnitt 3.6.4 in Jens

\hypertarget{rentenversicherung}{%
\subsubsection{Rentenversicherung}\label{rentenversicherung}}

3.6.5 in Jens

\hypertarget{hinterbliebenenrentenversicherung}{%
\subsubsection{Hinterbliebenenrentenversicherung}\label{hinterbliebenenrentenversicherung}}

3.6.6 in Jens

\hypertarget{berufs-erwerbsunfuxe4higkeitsversicherung}{%
\subsubsection{Berufs-/Erwerbsunfähigkeitsversicherung}\label{berufs-erwerbsunfuxe4higkeitsversicherung}}

3.6.7 Berufs-/Erwerbsunfähigkeitsversicherung in Jens

\hypertarget{fondsgebundene-lebens--oder-rentenversicherung}{%
\subsubsection{Fondsgebundene Lebens- oder
Rentenversicherung}\label{fondsgebundene-lebens--oder-rentenversicherung}}

3.6.8 Fondsgebundene Lebens- oder Rentenversicherung in Jens

\hypertarget{versicherungsmathematische-notation}{%
\subsubsection{Versicherungsmathematische
Notation}\label{versicherungsmathematische-notation}}

Versicherungsmathematische Notation

\hypertarget{elementare-finanzmathematik}{%
\subsection{Elementare
Finanzmathematik}\label{elementare-finanzmathematik}}

\hypertarget{zinsrechnung}{%
\subsubsection{Zinsrechnung}\label{zinsrechnung}}

\textbf{Zins:} Der Zins kann als Preis für die Übernahme eines Risikos
und/oder als Preis für das Leihen von Geld angesehen werden.

Grund für Zins:\\
1. \textbf{Kreditausfallrisiko}: Risiko, dass der Kunde zahlungsunfähig
wird\\
2. \textbf{Opportunitätskosten}: Kosten aufgrund der entgangenen
Möglichkeit, den als Kredit gewährten Betrag anderweitig zu investieren

Notation:\\

\(K_t\) Kapitalwachstumsfunktion

\(t\) Zeiteinheit (standardmässig in Jahren)

\textbf{p.a.} per annum

\(i\) Zinssatz p.a.

Wie werden bereits angefallene Zinsen behandelt?\\
1. \textbf{Zinseszinsrechnung}: bereits angefallene (und nicht
ausgezahlte) Zinsen in den Folgeperioden wieder mitverzinst\\
2. \textbf{einfache Verzinsung}: die anfallenden Zinsen schlichtweg
angesammelt werden

Wann werden die Zinsen gutgeschrieben?\\
1. \textbf{nachschüssigen Verzinsung}\\
2. \textbf{vorschüssige Verzinsung}

Modell:\\

Einfache Verzinsung: \(K_t = K_0 \cdot (1 + t\cdot i)\)

Zinseszins: \(K_t = K_0 \cdot (1 + i)^t\)

\hypertarget{wichtigste-definitionen}{%
\paragraph{Wichtigste Definitionen}\label{wichtigste-definitionen}}

\textbf{Diskontfaktor:} \(v = \frac{1}{1+i}\) Welcher Betrag muss zum
Zeitpunkt t = 0 investiert werden, damit sich nach einem Jahr der
normierte Betrag der Höhe 1 ergibt?

\textbf{Aufzinsungsfaktor:} \(r = 1+i\)

\textbf{effektiver Zinssatz:} (pro Zeiteinheit) \(i\), falls der
Einheitsbetrag 1 nach \(t\) Zeiteinheiten auf \((1+i)^t\)

Beispiel: Bei einem Investment, welches sich über t Jahre erstreckt,
werden in den einzelnen Jahren (i.d.R. unterschiedliche) Zinssätze
gewährt, i.e.~\(i_1,\dots,\, i_t\). Interessiert man sich nun für die
erzielte durchschnittliche Verzinsung, so sucht man den Effektivzinssatz
i für den genannten Zeitraum.

Fall 1: \textbf{Zinseszins}\\
In diesem Fall lautet der Ansatz \[
    K_0 \cdot (1+i_1) \cdot \dots \cdot (1+i_t) \overset{!}{=}  K_0 \cdot (1+i)^t
  \]

Den effektiven Zinssatz i erhält man somit als geometrisches Mittel der
Aufzinsungsfaktoren, i.e. \[
    1+i = \left( \prod_{j=1}^t (1+i_j) \right)^{\frac{1}{t}}
  \]

Fall 1: \textbf{Einfache Verzinsung}\\
In diesem Fall lautet der Ansatz \[
    K_0 \cdot (1 + i_1 \dots + i_t) \overset{!}{=}  K_0 \cdot (1 + t \cdot i)
  \]

Den effektiven Zinssatz i erhält man somit als arithmetisches Mittel der
Aufzinsungsfaktoren, i.e. \[
    i = \frac{1}{t} \sum_{j=1}^t i_j
  \]

\textbf{effektiver Diskontsatz:} (pro Zeiteinheit)
\(d = 1 - v = \frac{i}{1+i} = iv\)

Beachte: Die Angabe des Effektivzinssatzes i oder des
Effektivdiskontsatzes d zueinander äquivalent:\\
\[
  1+i = \left(\frac{1}{1+i}\right)^{-1} = \left(1-\frac{i}{1+i}\right)^{-1} = (1-d)^{-1}
\]

\hypertarget{nominelle-verzinsung}{%
\paragraph{Nominelle Verzinsung}\label{nominelle-verzinsung}}

Bislang wurde angenommen, dass die Zinsen nur einmal pro Zeiteinheit,
d.h. am Ende eines Jahres, fällig werden.

Im Folgenden entfallen auf eine Zeiteinheit (Jahr) k Zinsperioden,
welche jeweils eine Länge von \(\frac{1}{k}\) des Jahres haben. Folglich
werden k-Mal pro Jahr Zinsen (nachschüssig) fällig. Diese Art der
Zinsfälligkeit bezeichnet man als \textbf{unterjährige Verzinsung}.

Die nachfolgenden Definitionen gewährleisten, dass der Zinsertrag über
die gesamte Zeiteinheit mit der Summe der in den zugehörigen
Zinsperioden erfolgenden Zinszahlungen übereinstimmt.

Der \textbf{nominelle Zinssatz} oder auch \textbf{Nominalzinssatz}
\(i^{(k)}\) pro Zeiteinheit wird durch folgende Gleichung definiert: \[
  1 + i = \left(1+\frac{i^{(k)}}{k}\right)^k
\]

Anders ausgedrückt:\\
Ein nomineller Zinssatz von \(i^{(k)}\) p.a. wird in k gleich großen
Teilzahlungen (Raten) unterjährig ausgeschüttet. Er stimmt mit einem
effektiven Zinssatz von \(\frac{i^{(k)}}{k}\) pro \(\frac{1}{k}\) eines
Jahres überein.

Beachte: Für k = 1 stimmen effektiver und nomineller Zinssatz überein.

Analog existiert auch ein \textbf{nomineller Diskontsatz}, welcher mit
\(d^{(k)}\) bezeichnet und mittels des effektiven Diskontsatzes d
definiert wird, nämlich\\
\[
  1-d = \left(1-\frac{d^{(k)}}{k}\right)^k.
\] Und wegen \(1+i = (1-d)^{-1}\) gilt: \[
  \left(1+\frac{i^{(k)}}{k}\right)^{k} = \left(1-\frac{d^{(k)}}{k}\right)^{-k}
\]

\hypertarget{stetige-verzinsung}{%
\paragraph{Stetige Verzinsung}\label{stetige-verzinsung}}

4.1.6 Stetige Verzinsung

\hypertarget{zahlungsstruxf6me-cash-flows}{%
\subsubsection{Zahlungsströme (cash
flows)}\label{zahlungsstruxf6me-cash-flows}}

Die Mehrheit der finanziellen Zahlungsvorgänge, insbesondere alle hier
betrachteten Zahlungsschemata, lassen sich unter dem gemeinsamen Dach
eines Zahlungsstroms darstellen. Die Zeiteinheit beträgt, sofern nichts
anderes gesagt wird, wiederum 1 Jahr, und die Verzinsung erfolgt
nachschüssig (sowie ggf. unterjährig).

\textbf{(diskreter) Zahlungsstrom:} ist eine Serie von Zahlungen
\(z(t_j)\) , welche zu diskreten Zeitpunkten \(t_j\) mit
\(j \in \{0,\dots, J\}\) gemacht werden. Die Zahlungen \(z(t_j)\) können
positiv, negativ oder auch Null sein. In vielen Fällen ist \(t_j = j\),
so dass die Zahlungen äquidistant, d.h. in gleichen Abständen, über die
Zeitachse verteilt sind --- jedoch ist das nicht notwendigerweise der
Fall.

\hypertarget{bewertung-und-uxe4quivalenz-von-zahlungsstruxf6men}{%
\paragraph{Bewertung und Äquivalenz von
Zahlungsströmen}\label{bewertung-und-uxe4quivalenz-von-zahlungsstruxf6men}}

Auf Basis eines effektiven Zinssatzes von \(i\) p.a. wird der
\textbf{Wert des Zahlungsstroms} zu einem ausgewählten Zeitpunkt
\(t^{\ast}\, (t_0 \leq t^{\ast} \leq t_J)\) definiert als (vom
Englischen „cash flow``) \[
  CF(t^{\ast};i) = \sum_{j=0}^J z(t_j)\,\cdot\, (1+i)^{t^{\ast}-t_j}.
\]

Der \textbf{Barwert} (\textbf{present value}) des Zahlungsstroms zum
Zinssatz i, in Zeichen: \textbf{PV (i)} (vom Englischen „present
value``), ist der Wert des Zahlungsstroms zum Zeitpunkt
\(t^{\ast} = t_0\), d.h. \[
  PV(i) := CF(t_0;i) =  \sum_{j=0}^J z(t_j)\,\cdot\, (1+i)^{t_0-t_j}.
\]

Der \textbf{Endwert} (\textbf{terminal value}) des Zahlungsstroms zum
Zinssatz i, in Zeichen: \textbf{TV (i)} ist der Wert des Zahlungsstroms
zum Zeitpunkt \(t^{\ast} = t_J\), d.h. \[
  TV(i) := CF(t_J;i) =  \sum_{j=0}^J z(t_j)\,\cdot\, (1+i)^{t_J-t_j}.
\] Es gilt für drei Zeitpunkte \(s\), \(t\) und \(t^{\ast}\):
\begin{align*}
  CF(t;i) &= (1+i)^{t-s}\,\cdot\, CF(s;i)\\
  CF(t;i) &= (1+i)^{t-s}\,\cdot\, CF(s;i)\\
  CF(t^{\ast};i) &= (1+i)^{t^{\ast}}\,\cdot\,PV(i)\\
  TV(i) &= (1+i)^{t_J}\,\cdot PV(i)\\
  PV(i) &= v^{t_J}\,\cdot TV(i)
\end{align*}

\hypertarget{spezielle-zahlungsstruxf6me}{%
\paragraph{Spezielle Zahlungsströme}\label{spezielle-zahlungsstruxf6me}}

Für diese speziellen Bar- und Endwerte hat sich eine besondere Notation
entwickelt, welche sich auch in der Versicherungsmathematik in sehr
ähnlicher Form wiederfinden wird.

Die Zeiteinheit betrage weiterhin 1 Jahr, und der jährliche (effektive)
Zinssatz sei \(i\) p.a.\\
Die jährlichen Zahlungen haben jeweils die normierte Höhe 1 (sofern
nichts anderes gesagt wird).

\hypertarget{einmalzahlung}{%
\subparagraph{Einmalzahlung}\label{einmalzahlung}}

Eine \textbf{Einmalzahlung} ist eine Zahlung, die einmalig zu einem
vorab festgelegten Zeitpunkt geleistet wird.

Einmalzahlung nach \(n\) Jahren:\\
Zugehöriger Zahlungsstrom: \(J=n,\, j\in\{0,\dots,n\}\) mit \(z(t_j)=0\)
für \(t_j= 0,\dots,n-1\) und \(z(n)=1\).\\
Zugehöriger Barwert: \(PV(i) = CF(0;i) = v^n\)\\
Zugehöriger Endwert: \(TV(i) = CF(n;i) = 1\)

Einmalzahlung sofort:\\
Zugehöriger Zahlungsstrom: \(J=n,\, j\in\{0,\dots,n\}\) mit \(z(0)=1\)
und \(z(t_j)=0\) für \(t_j= 1,\dots,n\).\\
Zugehöriger Barwert: \(PV(i) = CF(0;i) = 1\)\\
Zugehöriger Endwert: \(TV(i) = CF(n;i) = (1+i)^n\)

\hypertarget{zeitrenten}{%
\subparagraph{Zeitrenten}\label{zeitrenten}}

Da die Zahlungen deterministisch sind, handelt es sich bei den folgenden
Zahlungsschemata um Zeitrenten.\\
Die Rentenzahlungen können entweder vorschüssig zu Beginn oder
nachschüssig, d.h. am Ende des jeweiligen Jahres, erfolgen.

(temporäre) Zeitrente, vorschüssig:\\
Eine \(n\) Jahre lang vorschüssig zahlbare (temporäre) Zeitrente (der
Höhe 1).\\
Zugehöriger Zahlungsstrom: \(J=n,\, j\in\{0,\dots,n\}\) mit \(z(t_j)=1\)
für \(t_j= 1,\dots,n-1\) und \(z(n)=1\).\\
Zugehöriger Barwert:
\(PV(i) = CF(0;i) = 1 + v + v^2 + \cdots + v^{n-1} = \frac{1-v^n}{1-v} = \frac{1-v^n}{d}\)\\
Dieser Barwert wird als \textbf{vorschüssiger Rentenbarwertfaktor}
bezeichnet. Symbol: \[
  \ddot{a}_{\tikz[baseline=(n.base)]{\node[inner sep=1pt,draw,rectangle] (n) {\( n \)};}} = \frac{1 - v^n}{d}
\]

\[
  \ax*{\angln}
\]

\end{document}
